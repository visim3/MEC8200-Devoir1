\documentclass[12pt, letterpaper]{article}
	\usepackage[utf8]{inputenc}
    \usepackage[T1]{fontenc}
    \usepackage[french]{babel}
% \usepackage[style=numeric]{biblatex}
%	\addbibresource{Tp1.bib}
\usepackage[margin=1.2in]{geometry}
\usepackage{graphicx}
\usepackage{chngcntr}
	\counterwithin{figure}{section}
\usepackage{amsmath}
\usepackage{float}
\usepackage{booktabs}
\usepackage{longtable}

\begin{document}
% ===============================================================
% Title Section
% ===============================================================
\pagenumbering{gobble}
\begin{titlepage}
	\centering
    {\scshape\Large
	MEC8200 : Mécanique  des fluides assisté par ordinateur \par
    }
    \vspace{4cm}
    {\scshape\LARGE
	Devoir 1 - Équations de la couche limite et de Navier-Stokes \par
	}
    \vspace{4cm}
    {\Large
	\begin{tabular}{ll}
		Anthony Rabenantoandro & 1570555 \\
		Vincent Simard & 1625589 \\
	\end{tabular}
	}
    \vfill
    {\Large
    15 février 2017
    }
\end{titlepage}
\newpage
% ===========================================================================
% Table of content and others
% ===========================================================================
\pagenumbering{roman}
\tableofcontents
\listoffigures
\listoftables
\newpage

% ===========================================================================
% Main section
% ===========================================================================
\pagenumbering{arabic}

\section{Introduction}
	\input{Content/0_Introduction.tex}
% Anthony
\section{Dévellopement des équations de Blasius}
	\input{Content/1_Blasius.tex}
% Vincent
\section{Résolution à $Re=100$}
	\input{Content/2_pl100.tex}
% Anthony
\section{Résolution à $Re=10\,00$}
	\input{Content/3_pl10000.tex}
% Vincent
\section{Résolution d'une plaque verticale}
		\input{Content/3_verti.tex}
% Anthony
\section{Conclusion}
	\input{Content/X_Conclusion.tex}

\clearpage

% ===========================================================================
% Annexes
% ===========================================================================
\appendix

\section{Données des coupe à $Re=100$}
	\begin{table}[h!]
\centering
	\caption{Coupe en $X = 0.5$}
	\begin{tabular}{llll}
	Points & Ux & Vy & Pression \\ 
\midrule 
	0.0 & 0.0 & 0.0 & -0.104273007 \\ 
	0.00872981239 & 0.0793529312 & -0.0137703102 & -0.103435616 \\ 
	0.018332606 & 0.11821685 & -0.0299171935 & -0.093850035 \\ 
	0.028895679 & 0.162921741 & -0.0297036553 & -0.0858368619 \\ 
	0.0405150593 & 0.205740482 & -0.0363522446 & -0.0809837416 \\ 
	0.0532963776 & 0.254330726 & -0.0345295998 & -0.0765186051 \\ 
	0.0673558277 & 0.304265426 & -0.0369649606 & -0.072989592 \\ 
	0.0828212229 & 0.360619581 & -0.034036508 & -0.0694843197 \\ 
	0.0998331576 & 0.418665806 & -0.0338219084 & -0.0662689632 \\ 
	0.118546286 & 0.482975584 & -0.0299768735 & -0.0628619907 \\ 
	0.139130727 & 0.548085094 & -0.0276953733 & -0.0594444791 \\ 
	0.161773612 & 0.618001122 & -0.0227951507 & -0.0556656414 \\ 
	0.186680785 & 0.686243112 & -0.0185391223 & -0.0516696094 \\ 
	0.214078676 & 0.755793353 & -0.0123401542 & -0.0471505773 \\ 
	0.244216356 & 0.8189276 & -0.00632227615 & -0.0422367279 \\ 
	0.277367804 & 0.877649433 & 0.000760168364 & -0.036657016 \\ 
	0.313834397 & 0.923903431 & 0.00730697059 & -0.0305573358 \\ 
	0.353947649 & 0.960626729 & 0.0135397566 & -0.0237173831 \\ 
	0.398072226 & 0.982320066 & 0.0183093529 & -0.0163084308 \\ 
	0.446609261 & 0.995727223 & 0.0216448548 & -0.00811315471 \\ 
	0.5 & 0.998222752 & 0.0236854229 & -0.00411595636 \\ 
\end{tabular}
	\label{tbl.100_X05}
\end{table}

\begin{table}[h!]
\centering
	\caption{Coupe en $X = 1.0$}
	\begin{tabular}{llll}
	Points & Ux & Vy & Pression \\ 
\midrule 
	0.0 & 0.0 & 0.0 & -0.0549115425 \\ 
	0.00872981239 & 0.0464252627 & -0.00264179415 & -0.0533211189 \\ 
	0.018332606 & 0.0975095622 & 0.00196275814 & -0.0528256356 \\ 
	0.028895679 & 0.15319637 & 0.000110472473 & -0.0523162679 \\ 
	0.0405150593 & 0.214483214 & 0.00555186346 & -0.0516278767 \\ 
	0.0532963776 & 0.280602382 & 0.00576356579 & -0.0509827034 \\ 
	0.0673558277 & 0.35270706 & 0.0128000824 & -0.0501208095 \\ 
	0.0828212229 & 0.429124707 & 0.0160602927 & -0.0492362853 \\ 
	0.0998331576 & 0.510540165 & 0.0255755223 & -0.0480788137 \\ 
	0.118546286 & 0.593786048 & 0.0325780356 & -0.0467647468 \\ 
	0.139130727 & 0.678358988 & 0.0449163284 & -0.0450536953 \\ 
	0.161773612 & 0.758972896 & 0.0552391773 & -0.0429889371 \\ 
	0.186680785 & 0.833895494 & 0.0690224865 & -0.0403550728 \\ 
	0.214078676 & 0.896554783 & 0.0799730296 & -0.0371835656 \\ 
	0.244216356 & 0.946319293 & 0.0914609791 & -0.0333508058 \\ 
	0.277367804 & 0.978795502 & 0.0987320742 & -0.0289592943 \\ 
	0.313834397 & 0.998097024 & 0.104344421 & -0.0240034051 \\ 
	0.353947649 & 1.00391935 & 0.106124696 & -0.0186242131 \\ 
	0.398072226 & 1.0038621 & 0.106591762 & -0.012822288 \\ 
	0.446609261 & 0.998371455 & 0.105410119 & -0.00664596813 \\ 
	0.5 & 0.996598573 & 0.103927805 & -0.00359419258 \\ 
\end{tabular}
	\label{tbl.100_X1}
\end{table}

%\begin{table}[h!]
\clearpage
\centering
	\caption{Coupe en $Y = 0.05$}
	\begin{longtable}{llll}
	Points & Ux & Vy & Pression \\ 
\midrule 
	-2.0 & 1.0 & 0.0 & 0.000562430539 \\ 
	-1.79763677 & 0.999845091 & 6.97324263e-05 & 0.000817737038 \\ 
	-1.6136702 & 0.999469641 & 0.000156356339 & 0.0012225013 \\ 
	-1.44642786 & 0.998689871 & 0.000283525655 & 0.0019522661 \\ 
	-1.29438937 & 0.997625191 & 0.00046359092 & 0.00308325522 \\ 
	-1.15617256 & 0.995916159 & 0.000727959954 & 0.00471598349 \\ 
	-1.03052092 & 0.9937584 & 0.00107673276 & 0.00695735297 \\ 
	-0.916292149 & 0.990662723 & 0.00155788395 & 0.00991535477 \\ 
	-0.812447815 & 0.98697935 & 0.00214932867 & 0.013683012 \\ 
	-0.718043875 & 0.982064886 & 0.00292834094 & 0.0183448591 \\ 
	-0.632222111 & 0.976506804 & 0.0038405547 & 0.0239613672 \\ 
	-0.554202325 & 0.969425376 & 0.005022458 & 0.0305924735 \\ 
	-0.483275248 & 0.961722865 & 0.00638179695 & 0.0382815221 \\ 
	-0.418796087 & 0.952129479 & 0.00817179183 & 0.0470953265 \\ 
	-0.360178667 & 0.941943164 & 0.0102655449 & 0.0571072017 \\ 
	-0.306890104 & 0.929277858 & 0.0131428059 & 0.0684395213 \\ 
	-0.258445956 & 0.915915881 & 0.0166765107 & 0.0812654404 \\ 
	-0.214405821 & 0.899060415 & 0.0218442934 & 0.0957965109 \\ 
	-0.174369335 & 0.881063296 & 0.028664378 & 0.112374384 \\ 
	-0.137972529 & 0.85783498 & 0.039571052 & 0.131170753 \\ 
	-0.104884524 & 0.832913046 & 0.0553476281 & 0.152282478 \\ 
	-0.0748045194 & 0.800041065 & 0.0838304044 & 0.174095801 \\ 
	-0.0474590607 & 0.775710056 & 0.133425935 & 0.18659973 \\ 
	-0.0225995527 & 0.766852459 & 0.188852103 & 0.174869 \\ 
	0.0 & 0.771238765 & 0.204167112 & 0.149231321 \\ 
	0.00966844026 & 0.761614307 & 0.195172336 & 0.109094514 \\ 
	0.0208850819 & 0.752698377 & 0.175877275 & 0.0811777861 \\ 
	0.0338978375 & 0.720718018 & 0.147683796 & 0.053845317 \\ 
	0.0489943177 & 0.685453347 & 0.120268078 & 0.0299003928 \\ 
	0.066508188 & 0.63580787 & 0.0933151159 & 0.0104106591 \\ 
	0.0868265436 & 0.589709071 & 0.0737501282 & -0.00480568711 \\ 
	0.110398465 & 0.539428519 & 0.0556410318 & -0.0166617055 \\ 
	0.137744944 & 0.49514636 & 0.0441507675 & -0.0257663881 \\ 
	0.169470397 & 0.450998652 & 0.0327912934 & -0.0328232732 \\ 
	0.206276028 & 0.412405176 & 0.0265457194 & -0.0381963727 \\ 
	0.248975322 & 0.37548992 & 0.0192185505 & -0.0423270458 \\ 
	0.298512028 & 0.343131256 & 0.0160915901 & -0.0454137949 \\ 
	0.355981016 & 0.313071967 & 0.0109518957 & -0.0477879241 \\ 
	0.422652478 & 0.286890535 & 0.00967574584 & -0.049616071 \\ 
	0.5 & 0.263549858 & 0.0057089665 & -0.0511490974 \\ 
	0.577347522 & 0.246555 & 0.00555759881 & -0.0524811137 \\ 
	0.644018984 & 0.23543266 & 0.00273935373 & -0.0539423976 \\ 
	0.701487972 & 0.228563348 & 0.00323708325 & -0.0554367641 \\ 
	0.751024678 & 0.224040996 & 0.000982328649 & -0.0571101646 \\ 
	0.793723972 & 0.221840184 & 0.00151408216 & -0.0588402915 \\ 
	0.830529603 & 0.22059089 & -0.000621815075 & -0.0607017965 \\ 
	0.862255056 & 0.221102788 & -0.00034034159 & -0.0625847848 \\ 
	0.889601535 & 0.221921935 & -0.00283346397 & -0.0645795222 \\ 
	0.913173456 & 0.224508967 & -0.00342589761 & -0.0666060201 \\ 
	0.933491812 & 0.227054823 & -0.00727862041 & -0.0687811729 \\ 
	0.951005682 & 0.231355998 & -0.0100985169 & -0.0710344668 \\ 
	0.966102162 & 0.234681639 & -0.0167452618 & -0.0733825757 \\ 
	0.979114918 & 0.238626356 & -0.0222963557 & -0.0754501834 \\ 
	0.99033156 & 0.240150345 & -0.0302281629 & -0.0770168177 \\ 
	1.0 & 0.241799014 & -0.0349996707 & -0.0776701902 \\ 
	1.01002919 & 0.241731522 & -0.0412999724 & -0.077325912 \\ 
	1.0210613 & 0.242259914 & -0.0440936505 & -0.0755617815 \\ 
	1.03319662 & 0.242942339 & -0.0466655639 & -0.072500087 \\ 
	1.04654548 & 0.245499162 & -0.0453225696 & -0.068180385 \\ 
	1.06122922 & 0.250109609 & -0.0438764324 & -0.0632005449 \\ 
	1.07738133 & 0.256735922 & -0.040542677 & -0.0578975092 \\ 
	1.09514865 & 0.265093848 & -0.0379781876 & -0.0526171068 \\ 
	1.11469271 & 0.274606132 & -0.0347347609 & -0.0474319545 \\ 
	1.13619117 & 0.285255337 & -0.0322727006 & -0.0424149336 \\ 
	1.15983948 & 0.29660457 & -0.0295789219 & -0.0375736222 \\ 
	1.18585262 & 0.308762373 & -0.0274869794 & -0.0329187408 \\ 
	1.21446708 & 0.321451354 & -0.0253226834 & -0.0284537601 \\ 
	1.24594298 & 0.334802875 & -0.0235931906 & -0.0241846927 \\ 
	1.28056646 & 0.348648017 & -0.0218435518 & -0.0201237605 \\ 
	1.3186523 & 0.363101681 & -0.0204061224 & -0.016284867 \\ 
	1.36054672 & 0.378060391 & -0.0189580468 & -0.0126896382 \\ 
	1.40663059 & 0.393609355 & -0.0177369251 & -0.0093601138 \\ 
	1.45732284 & 0.409678072 & -0.0164970056 & -0.00632384745 \\ 
	1.51308431 & 0.426315316 & -0.0154249427 & -0.00360636049 \\ 
	1.57442194 & 0.443457177 & -0.014319737 & -0.00123450485 \\ 
	1.64189332 & 0.461108798 & -0.01334185 & 0.000771009427 \\ 
	1.71611184 & 0.479190593 & -0.0123172305 & 0.00239378004 \\ 
	1.79775222 & 0.497656653 & -0.011394505 & 0.00362883303 \\ 
	1.88755663 & 0.516395949 & -0.0104179119 & 0.00448124718 \\ 
	1.98634149 & 0.535309081 & -0.00953169338 & 0.00497183721 \\ 
	2.09500483 & 0.55425078 & -0.0085957413 & 0.00513434114 \\ 
	2.2145345 & 0.5730783 & -0.00775042603 & 0.00501793472 \\ 
	2.34601714 & 0.591629035 & -0.0068731179 & 0.00468115714 \\ 
	2.49064805 & 0.609743497 & -0.00609233057 & 0.00419036596 \\ 
	2.64974205 & 0.627277143 & -0.00530881738 & 0.00361054298 \\ 
	2.82474544 & 0.644088985 & -0.00462253789 & 0.00300147989 \\ 
	3.01724918 & 0.660091638 & -0.00396857569 & 0.00240863055 \\ 
	3.22900329 & 0.675187101 & -0.00339607284 & 0.0018576809 \\ 
	3.46193281 & 0.689387207 & -0.00290059966 & 0.00133846001 \\ 
	3.71815528 & 0.702694002 & -0.0024607675 & 0.00087194953 \\ 
	4.0 & 0.714852191 & -0.00198672436 & 0.000668441941 \\ 
\end{longtable}
	\label{tbl.100_Y005}
%\end{table}
	
% Vincent
	
\clearpage
\section{Données des coupe à $Re=10\,000$}
	\begin{table}[h!]
\centering
	\caption{Coupe en $X = 0.5$}
	\begin{tabular}{llll}
	Points & Ux & Vy & Pression \\ 
\midrule 
	0.0 & 0.0 & 0.0 & -0.0111047813 \\ 
	0.00872981239 & 0.371919544 & -0.152768421 & -0.010723657 \\ 
	0.018332606 & 0.695444497 & -0.0697721587 & -0.0111195365 \\ 
	0.028895679 & 0.971931543 & -0.0987785 & -0.00963862545 \\ 
	0.0405150593 & 0.991659885 & 0.0171687987 & -0.00888528891 \\ 
	0.0532963776 & 1.01668245 & -0.020033704 & -0.00855286568 \\ 
	0.0673558277 & 1.00767027 & 0.0340559743 & -0.00831858128 \\ 
	0.0828212229 & 1.00726561 & -0.0110262566 & -0.00804126734 \\ 
	0.0998331576 & 1.01150493 & 0.0293722133 & -0.00768267759 \\ 
	0.118546286 & 1.00234132 & -0.00389263319 & -0.00729667398 \\ 
	0.139130727 & 1.01316749 & 0.0234999483 & -0.0068459793 \\ 
	0.161773612 & 0.999443448 & 0.00126660601 & -0.00636738487 \\ 
	0.186680785 & 1.01399148 & 0.0186491514 & -0.00583210361 \\ 
	0.214078676 & 0.996575835 & 0.00478288273 & -0.00526502879 \\ 
	0.244216356 & 1.01464682 & 0.0149931504 & -0.00464580285 \\ 
	0.277367804 & 0.993205153 & 0.007020143 & -0.00399135503 \\ 
	0.313834397 & 1.01559909 & 0.0123922434 & -0.00328613569 \\ 
	0.353947649 & 0.989199653 & 0.00833385304 & -0.00254072909 \\ 
	0.398072226 & 1.01641084 & 0.0106941066 & -0.0017393905 \\ 
	0.446609261 & 0.986209694 & 0.00901582367 & -0.000884091484 \\ 
	0.5 & 1.01207041 & 0.00998949588 & -0.000459299998 \\ 
\end{tabular}
	\label{tbl.10k_X05}
\end{table}

\begin{table}[h!]
\centering
	\caption{Coupe en $X = 1.0$}
	\begin{tabular}{llll}
	Points & Ux & Vy & Pression \\ 
\midrule 
	0.0 & 0.0 & 0.0 & -0.0229987194 \\ 
	0.00872981239 & 0.291089793 & 0.120070903 & -0.0259807228 \\ 
	0.018332606 & 0.633038959 & -0.0472205391 & -0.0252629499 \\ 
	0.028895679 & 0.847073263 & 0.0234063115 & -0.02188209 \\ 
	0.0405150593 & 0.988449771 & -0.0226506905 & -0.0192959628 \\ 
	0.0532963776 & 1.01289439 & 0.00753902382 & -0.0166938878 \\ 
	0.0673558277 & 1.01760026 & -0.0152506344 & -0.0146407361 \\ 
	0.0828212229 & 1.01456059 & 0.00278800793 & -0.0128692647 \\ 
	0.0998331576 & 1.01179157 & -0.0110336302 & -0.0114087354 \\ 
	0.118546286 & 1.01214206 & 0.000675932923 & -0.0101011533 \\ 
	0.139130727 & 1.00769318 & -0.00788391027 & -0.00897287792 \\ 
	0.161773612 & 1.01029217 & -0.000315400814 & -0.00791416353 \\ 
	0.186680785 & 1.00517737 & -0.00543423658 & -0.00697239183 \\ 
	0.214078676 & 1.00827867 & -0.000691416346 & -0.00604849848 \\ 
	0.244216356 & 1.00377854 & -0.00354071366 & -0.00520316513 \\ 
	0.277367804 & 1.00560134 & -0.000733307252 & -0.00434029283 \\ 
	0.313834397 & 1.00347545 & -0.00212688339 & -0.00352668589 \\ 
	0.353947649 & 1.00203816 & -0.000638755844 & -0.00266613719 \\ 
	0.398072226 & 1.00367604 & -0.00117149665 & -0.00182504014 \\ 
	0.446609261 & 0.999277784 & -0.000570109055 & -0.000904131947 \\ 
	0.5 & 0.998944401 & -0.000769328627 & -0.000486150916 \\ 
\end{tabular}
	\label{tbl.10k_X1}
\end{table}

%\begin{table}[h!]
\clearpage
\centering
	\caption{Coupe en $Y = 0.05$}
	\begin{longtable}{llll}
	Points & Ux & Vy & Pression \\ 
\midrule 
	-2.0 & 1.0 & 0.0 & 0.000308635662 \\ 
	-1.79763677 & 0.997361089 & -0.000398843371 & 0.000208134785 \\ 
	-1.6136702 & 1.00415898 & 0.000938961701 & 0.000186088936 \\ 
	-1.44642786 & 0.996271486 & -0.00112327077 & 0.000344328638 \\ 
	-1.29438937 & 1.00633782 & 0.00173043066 & 0.00044606246 \\ 
	-1.15617256 & 0.994254777 & -0.00155948424 & 0.000682736079 \\ 
	-1.03052092 & 1.0080704 & 0.00226197921 & 0.000943464075 \\ 
	-0.916292149 & 0.991478395 & -0.0015741017 & 0.0013049062 \\ 
	-0.812447815 & 1.00962723 & 0.00257186524 & 0.00174919785 \\ 
	-0.718043875 & 0.987935391 & -0.00119367006 & 0.00226992093 \\ 
	-0.632222111 & 1.01102273 & 0.00282744798 & 0.00289975233 \\ 
	-0.554202325 & 0.983754638 & -0.00056545732 & 0.00359224781 \\ 
	-0.483275248 & 1.0121285 & 0.00324118526 & 0.00439798969 \\ 
	-0.418796087 & 0.979162594 & 8.20963475e-05 & 0.00527428195 \\ 
	-0.360178667 & 1.01268033 & 0.00399400694 & 0.00628231359 \\ 
	-0.306890104 & 0.974312226 & 0.000496895379 & 0.00742198058 \\ 
	-0.258445956 & 1.01219335 & 0.00525646197 & 0.00878631272 \\ 
	-0.214405821 & 0.969025191 & 0.000595594488 & 0.0104387194 \\ 
	-0.174369335 & 1.00982963 & 0.00750740932 & 0.0125317124 \\ 
	-0.137972529 & 0.962575346 & 0.00107630764 & 0.0151337959 \\ 
	-0.104884524 & 1.00458565 & 0.0126484207 & 0.0183490729 \\ 
	-0.0748045194 & 0.954250622 & 0.0050084681 & 0.0217873753 \\ 
	-0.0474590607 & 0.997866025 & 0.0278977723 & 0.0232208102 \\ 
	-0.0225995527 & 0.95546596 & 0.0198347861 & 0.0199331637 \\ 
	0.0 & 1.00930982 & 0.0465636597 & 0.0146688281 \\ 
	0.00966844026 & 0.972273992 & 0.0191884749 & 0.00776246274 \\ 
	0.0208850819 & 1.01105878 & 0.0534242737 & 0.0037094413 \\ 
	0.0338978375 & 0.98745353 & 0.0115110256 & 0.000480709736 \\ 
	0.0489943177 & 1.01102177 & 0.0500612812 & -0.00151825475 \\ 
	0.066508188 & 0.99592902 & 0.00557278876 & -0.00261396894 \\ 
	0.0868265436 & 1.00870546 & 0.045038617 & -0.00326167204 \\ 
	0.110398465 & 1.00020938 & 0.00262659131 & -0.00370065184 \\ 
	0.137744944 & 1.00627899 & 0.0411633422 & -0.00424174994 \\ 
	0.169470397 & 1.00385554 & 0.00171240636 & -0.00490443624 \\ 
	0.206276028 & 1.0044366 & 0.0359200164 & -0.00503027316 \\ 
	0.248975322 & 1.00538074 & 0.00347112929 & -0.0063735371 \\ 
	0.298512028 & 1.0075225 & 0.0297152065 & -0.00616463501 \\ 
	0.355981016 & 1.00172021 & 0.00206913528 & -0.00665904839 \\ 
	0.422652478 & 1.01359036 & 0.0333321151 & -0.0101729695 \\ 
	0.5 & 1.01022899 & -0.0104389585 & -0.0086385996 \\ 
	0.577347522 & 1.00077173 & 0.0296487073 & -0.0079139131 \\ 
	0.644018984 & 1.01801394 & -0.00202678704 & -0.0117976366 \\ 
	0.701487972 & 1.00355009 & 0.0148198625 & -0.0115288734 \\ 
	0.751024678 & 1.00953233 & -0.00229367225 & -0.0110314556 \\ 
	0.793723972 & 1.00574401 & 0.0127119433 & -0.0117196548 \\ 
	0.830529603 & 1.00599132 & -0.00540443081 & -0.011606968 \\ 
	0.862255056 & 1.00535015 & 0.0118289652 & -0.0121726952 \\ 
	0.889601535 & 1.00528373 & -0.00701779383 & -0.0125698451 \\ 
	0.913173456 & 1.00498769 & 0.00978867403 & -0.013137754 \\ 
	0.933491812 & 1.00527165 & -0.00762400304 & -0.0137699369 \\ 
	0.951005682 & 1.00543626 & 0.00720910128 & -0.0144498066 \\ 
	0.966102162 & 1.00580067 & -0.00826597982 & -0.0152086213 \\ 
	0.979114918 & 1.00632218 & 0.00429434874 & -0.0160668884 \\ 
	0.99033156 & 1.00728262 & -0.00952273497 & -0.0168318583 \\ 
	1.0 & 1.00658998 & -0.000247082078 & -0.0173649783 \\ 
	1.01002919 & 1.00798626 & -0.0112896355 & -0.0173837871 \\ 
	1.0210613 & 1.00537018 & -0.00633675967 & -0.0166203608 \\ 
	1.03319662 & 1.00571067 & -0.0141312068 & -0.0150530547 \\ 
	1.04654548 & 1.0015929 & -0.011395206 & -0.0128392353 \\ 
	1.06122922 & 1.00076326 & -0.0154592224 & -0.0104146418 \\ 
	1.07738133 & 0.996975552 & -0.0130551032 & -0.00809391309 \\ 
	1.09514865 & 0.99614234 & -0.0146985296 & -0.00601945448 \\ 
	1.11469271 & 0.993300022 & -0.012615732 & -0.00421118928 \\ 
	1.13619117 & 0.992603852 & -0.0130993261 & -0.00265726535 \\ 
	1.15983948 & 0.990514731 & -0.0112618833 & -0.00134895517 \\ 
	1.18585262 & 0.989970648 & -0.011173699 & -0.000267521807 \\ 
	1.21446708 & 0.988436351 & -0.00954179355 & 0.000599694712 \\ 
	1.24594298 & 0.988018685 & -0.00920723807 & 0.00127356565 \\ 
	1.28056646 & 0.986858956 & -0.00778251447 & 0.00177235677 \\ 
	1.3186523 & 0.986497895 & -0.00739097901 & 0.00211888451 \\ 
	1.36054672 & 0.985541339 & -0.00617775265 & 0.00233374744 \\ 
	1.40663059 & 0.985139915 & -0.00582816427 & 0.00243842695 \\ 
	1.45732284 & 0.984225513 & -0.00481570074 & 0.00245186438 \\ 
	1.51308431 & 0.983674365 & -0.00454747841 & 0.00239192769 \\ 
	1.57442194 & 0.982650087 & -0.00371154318 & 0.00227426909 \\ 
	1.64189332 & 0.981832699 & -0.003530624 & 0.00211313403 \\ 
	1.71611184 & 0.98055253 & -0.00284181118 & 0.00192135671 \\ 
	1.79775222 & 0.979350243 & -0.00273975445 & 0.00171059689 \\ 
	1.88755663 & 0.977680145 & -0.00217023383 & 0.00149160857 \\ 
	1.98634149 & 0.975988172 & -0.00213511399 & 0.00127378964 \\ 
	2.09500483 & 0.973827299 & -0.00166096828 & 0.00106553536 \\ 
	2.2145345 & 0.971582597 & -0.00168164707 & 0.000873104911 \\ 
	2.34601714 & 0.968893555 & -0.00128179273 & 0.00070154527 \\ 
	2.49064805 & 0.966103244 & -0.00134893138 & 0.000553015342 \\ 
	2.64974205 & 0.962936474 & -0.00100280333 & 0.000428703225 \\ 
	2.82474544 & 0.959691512 & -0.0011099298 & 0.000327074293 \\ 
	3.01724918 & 0.956190452 & -0.000795586467 & 0.000245705304 \\ 
	3.22900329 & 0.952655898 & -0.000941527182 & 0.000181119516 \\ 
	3.46193281 & 0.949042424 & -0.000635411884 & 0.00012605376 \\ 
	3.71815528 & 0.945431942 & -0.000827324891 & 7.25801824e-05 \\ 
	4.0 & 0.942007833 & -0.000511250955 & 4.57629841e-05 \\ 
\end{longtable}
	\label{tbl.10k_Y005}
%\end{table}
% Vincent

% Figure
% Vincent
\end{document}